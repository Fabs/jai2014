\newpage{}

\bibliographystyle{plainnat}
\bibliography{freesoft}


\newpage{}

\pagenumbering{roman} % para os curriculos

\setcounter{page}{1} %



\part*{Curriculum resumido dos autores}

O capítulo de livro será escrito em conjunto por quatro autores. O
Prof. Dr. Fabio Kon do IME/USP irá coordenar as atividades de escrita
e fará a revisão geral do texto. O conteúdo será elaborado pelo Prof.
Fabio Kon, pelos doutorandos do IME/USP Nelson Posse Lago e Paulo
Meirelles e pela mestranda do IME/USP Vanessa Sabino. Os quatro autores
possuem longa experiência com o tópico de software livre e estão atualmente
trabalhando ativamente na pesquisa do tópico específico do minicurso
aqui proposto.


\paragraph{Fabio Kon}

é Professor Associado do Departamento de Ciência da Computação da
Universidade de São Paulo.
%
Em 2000, recebeu grau de PhD em Ciência da Computação pela
Universidade de Illinois em Urbana-Champaign na área de Configuração
Automática de Sistemas Distribuídos baseados em Componentes.
%
Seus interesses de pesquisa incluem Software Livre, Sistemas
Distribuídos, Computação Musical, Tele-Saúde, Metodologias Ágeis de
Desenvolvimento de Software e Grades Computacionais.
%
Ele é membro da SBC, ACM e do Hillside Group e tem participado de
comitês de programa de conferências internacionais e workshops como
ACM OOPSLA, ACM/IFIP/USENIX Middleware, IEEE Agile, ACM GPCE, DOA, RAM
e MGC e de eventos nacionais tais como SBRC, SBCM, SugarLoafPLoP,
WebMídia e ESELAW.

Fabio publicou mais de 85 artigos em conferências internacionais e
nacionais como a ACM/IFIP/USENIX Middleware, ACM SAC, USENIX COOTS,
DOA, XP, SBRC, SBCM, WSL e WebMídia e em periódicos como
Communications of the ACM, Software: Practice \& Experience,
Concurrency: Practice \& Experience, Journal of the Brazilian Computer
Society, Computer Communications e IEEE Concurrency.

Nos últimos 5 anos, tem atuado como coordenador em vários projetos de
pesquisa e desenvolvimento de software livre financiados por agências
de pesquisa tais como CNPq, CAPES e FAPESP.
%
Esses projetos incluem, entre outros, esforços inovadores como o
InteGrade (\url{http://integrade.incubadora.fapesp.br}) um middleware
livre para Grades Computacionais, AcMus
(\url{http://gsd.ime.usp.br/acmus}) um software livre para medir,
analisar e simular a acústica de ambientes para prática musical, já
utilizado por vários grupos ao redor do mundo e Borboleta
(\url{http://ccsl.ime.usp.br/borboleta}), um software livre na área de
Tele-Saúde voltado para programas do SUS (Sistema Único de Saúde)
brasileiro.
%
Fabio atuou também como consultor em desenvolvimento de software,
baseado em software livre, para a Assembleia Legislativa do Estado de
São Paulo e assessora a Pró-Reitoria de Pós-Graduação da USP no
desenvolvimento de seu novo sistema de gerenciamento da pós-graduação,
baseado em software livre.


\paragraph{Nelson Posse Lago}

é Gerente Técnico do Centro de Competência em Software Livre do
IME/USP, onde trabalha pela divulgação do software livre, além de
doutorando do Departamento de Ciência da Computação da Universidade de
São Paulo.
%
É graduado em música pela Escola de Comunicações e Artes da USP e
mestre em ciência da computação pelo IME/USP (como bolsista CAPES),
onde desenvolveu um software livre para o processamento de áudio
distribuído em ambiente linux.

Durante o mestrado, participou do Programa de Aperfeiçoamento de
Ensino (PAE), onde ministrou aulas e participou do processo de
reestruturação da disciplina ``Introdução à computação''.
%
Ministrou diversos cursos e palestras sobre linux e software livre,
abordando tanto aspectos técnicos quanto conceituais.
%
Tem artigos publicados em eventos no Brasil e no exterior.
%
Participou do processo de criação da ONG
\textquotedbl{}LinuxSP\textquotedbl{}, onde ofereceu, juntamente com
outros voluntários, apoio técnico ao projeto dos telecentros da
prefeitura de São Paulo.

Atuou como administrador de sistemas em um dos primeiros provedores
comerciais de acesso à Internet em São Paulo (That's Internet!),
totalmente baseado em software livre.
%
Trabalhou como administrador de sistemas no provedor de acesso de
médio porte Orolix, onde participou do projeto e implantação das
infraestruturas de rede e software (30 servidores), e atuou como
consultor interno para o processo de desenvolvimento, sempre com base
em software livre. 


\paragraph{Paulo Meirelles}

é doutorando do Departamento de Ciência da Computação da Universidade
de São Paulo na área de Software Livre, com o projeto MANGUE: Métricas
e Ferramentas para Avaliação Automática da Qualidade de Software de
Livre, projeto aprovado pelo CNPq.
%
É Mestre em Computação pela Universidade Federal do Rio Grande do Sul
(UFRGS), na área de Sistema Embarcados e graduado em Tecnologia em
Desenvolvimento de Software pelo Centro Federal de Educação
Tecnológica do Rio Grande do Norte (CEFET-RN).
%
Tem 5 publicações, somando trabalhos em revistas, congressos e
workshops locais, nacionais e internacionais.

Possui experiência profissional na área de Ciência da Computação, com
ênfase em Engenharia de Software, Linguagens de Programação e Banco de
Dados.
%
Participou do desenvolvimento de 14 sistemas de software, a maioria
quando foi programador e analista no Tribunal de Contas do Estado do
Rio Grande do Norte.

Participou da criação do Projeto Software Livre do Rio Grande do
Norte, do qual ainda é membro ativo, colaborando em projetos de
disseminação do Software Livre e seus aspectos legais entre outros.
%
Desde 2006 faz parte da Associação Software Livre (ASL) e é membro da
organização do Fórum Internacional de Software Livre (FISL), um dos
maiores eventos do mundo sobre o tema, tendo atuado, nas duas últimas
edições, como um dos coordenadores da programação do FISL e da Arena
de Programação Livre.
%
Foi um dos coordenadores de produção e conteúdo do ciclo de debates
``Além das Redes de Colaboração: diversidade cultural e as tecnologias
do poder'', realizado no segundo semestre de 2007 em Porto Alegre e
Natal, promovido pela da Casa de Cinema de Porto Alegre, pela
Associação Software Livre e pelo Projeto Software Livre do Rio Grande
do Norte, dentro do programa Cultura e Pensamento do Ministério da
Cultura.
%
O conteúdo dos debates gerou o livro ``Além das Redes de Colaboração:
internet, diversidade cultural e as tecnologias do poder'', publicado
pela Editora da Universidade Federal da Bahia.


\paragraph{Vanessa Sabino}

é mestranda do Departamento de Ciência da Computação da Universidade
de São Paulo.
%
Tem graduação em Administração de Empresas pela Fundação Getúlio
Vargas e em Matemática Aplicada e Computacional pela Universidade de
São Paulo.
%
Atualmente desenvolve uma pesquisa sobre licenças de software livre,
visando classificá-las e levantar as características que influem na
compatibilidade entre as diversas licenças, de forma a auxiliar
criadores de software livre a escolher as licenças para seus projetos
e entender as limitações a que estão sujeitos.
%
É usuária de software livre desde 1997 e trabalha com desenvolvimento
de sistemas há dez anos.
%
Já participou de diversos projetos profissionalmente, tanto
brasileiros como internacionais, atuando em várias áreas de negócio e
adquirindo experiência em programação desde sistemas web e aplicações
desktop até jogos para celulares.
%
Também participa de projetos de software livre em seu tempo livre. Foi
palestrante no Fórum Internacional de Software Livre (FISL) por
três anos consecutivos, na LinuxConf e em diversos eventos sobre
a tecnologia Java.

\end{document}
