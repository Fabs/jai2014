\begin{resumo}
Esse cap�tulo tem como objetivo principal apresentar tanto os conceitos b�sicos relacionados ao empreendedorismo com foco em inova��o tecnol�gica como seu emprego em empresas startup de software e internet. O objetivo � incentivar o esp�rito empreendedor em alunos e profissionais de Computa��o e demais interessados em inova��o tecnol�gica em torno de software, apresentando o estado da arte de t�cnicas para a cria��o de empresas �geis, que entregam solu��es escal�veis a um grande n�mero de clientes. Ap�s ler o cap�tulo, o leitor deve ser capaz de refletir sobre as possibilidades de cria��o de um neg�cio sustent�vel a partir de uma ideia tecnol�gica inovadora e ter uma vis�o dos passos a seguir para criar uma empresa de sucesso. Abordamos aqui as t�cnicas de Startup Enxuta (\emph{Lean Startup}), Desenvolvimento de Cliente (\emph{Customer Development}) e Painel de Modelo de Neg�cios (\emph{Business Model Canvas}) e damos uma vis�o geral sobre o empreendedorismo de base tecnol�gica e o ecossistema de startups digitais.

\end{resumo}

\begin{abstract}
This chapter presents both the basic concepts in the area of entrepreneurship with focus on technological innovation and their application to internet and software startups. Its goal is to inspire the entrepreneurial spirit of Computer Science students, professionals, and anyone interested in software-related technological innovation. We present the state of the art on how to create agile companies that deliver scalable solutions to a large number of customers. After reading the chapter, the reader will be able to reflect upon the possibilities of creating a sustainable business from an innovative technological idea and will have an outline of the steps to follow to create a successful company. The chapter covers techniques from Lean Startup, Customer Development, and the Business Model Canvas and presents an overview of digital entrepreneurship and software startup ecosystems.

\end{abstract}
