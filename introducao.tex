\section{Introdução}

%\subsection{Inovação Tecnológica, Empreendedorismo e Startups}
responsável: Fabio revisor: Fabs

Ao longo do Século 20, boa parte da inovação na indústria de alta tecnologia foi desenvolvida em médias e grandes organizações, sejam elas corporações privadas ou laboratórios de pesquisa governamentais, civis ou militares \cite{Chesbrough2008}. No final do século, a partir da inovadora legislação americana, o \emph{Bayh–Dole Act} de 1980, e leis similares em outros países \cite{Mowery2004}, um arcabouço legal para transferência de tecnologia de universidades e institutos de pesquisas públicos para a industria de alta tecnologia foi estabelecido. Isso promoveu um aumento na velocidade com que os avanços científicos eram transformados em produtos tecnológicos e negócios economicamente viáveis. Contudo, com o grande crescimento e a popularização da Internet na década de 1990 e das tecnologias móveis na década de 2000, a velocidade de criação e adoção de novas tecnologias aumentou ainda mais. 

Os ciclos de desenvolvimento, da concepção da ideia à comercialização do produto, demoram tipicamente de 5 a 10 anos em grandes empresas \cite{ries:2011} e se dão por meio dos mecanismos tradicionais de transferência de tecnologia; este tempo é necessário dada a inerente burocracia e falta de agilidade associadas a grandes organizações estruturadas hierarquicamente.
Por outro lado, na era da Internet, ideias inovadoras podem ser concebidas, implementadas, testadas e comercializadas em 1 ou 2 anos ou, em alguns casos extremos, numa questão de alguns poucos meses \cite{benkler:2006,goldman:2005}. Após 5 ou 10 anos, muitas vezes, essas tecnologias são substituídas por uma geração mais nova e se tornam obsoletas.

Dessa forma, os mecanismos tradicionais para inovação tecnológica largamente utilizados no Século 20 muitas vezes não são apropriados para uma grande gama de novas ideias tecnológicas relacionadas à computação, software e à internet.

Startups de Software oferecem um ambiente muito mais ágil para a concepção e desenvolvimento de ideias inovadoras a um custo mais baixo. Uma startup é uma organização temporária em busca de um modelo de negócios rentável, repetitível e escalável
\cite{Blank2012}. Uma pequena startup fundada por dois ou três empreendedores e alguns poucos funcionários podem produzir e testar a viabilidade de dezenas de possibilidades para uma nova ideia de negócio, desenvolvendo um produto viável em questão de poucos meses. Recentemente, essa agilidade tem levado à criação de milhares de startups de software ao redor do mundo todos os anos. De acordo com o maior banco de dados de startups disponível \cite{crunchbase:2014}, mais de 200 mil startups foram fundadas nos últimos 10 anos. Essas iniciativas tem se concentrado em alguns poucos centros regionais onde um ecossistema de suporte ao empreendedorismo tecnológico floresceu. Indiscutivelmente, o Vale do Silício na California é o principal centro mundial de startups de software tendo sido o berço da maioria das principais empresas de TI da atualidade. Em seguida, centros como Tel-Aviv em Israel, Nova Iorque e Boston nos EUA e Londres e Paris na Europa também se destacam. O Brasil ainda está bem atrás destes grandes centros, mas cidades como São Paulo, Belo Horizonte, Recife, Rio de Janeiro, Porto Alegre e Campinas possuem um bom potencial para tornarem-se centros significativos para startups ao longo da próxima década.

No Século 21, a capacidade de um país desenvolver nova ciência e transformá-la em tecnologias inovadoras que levem a negócios sustentáveis, gerando lucro e empregos de alta qualidade é de fundamental importância para o desenvolvimento econômico e social  para a qualidade de vida de seus cidadãos \cite{unece:2012}. Com este minicurso, esperamos contribuir para difundir os conceitos básicos e as principais técnicas relacionadas ao empreendedorismo digital e incentivar jovens entre 20 e 70 anos a trilhar o caminho do empreendedorismo.


\paragraph{Empreendedorismo em grandes empresas}
responsável: Fabs revisor: Fabio

\paragraph{Empreendedorismo no governo e ONGs}
responsável: revisor: Fabio

\paragraph{Empreendedorismo social}
responsável: revisor: Fabio
