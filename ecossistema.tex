\section{Ecossistema}
respons�vel: Fabio revisor:

Ao longo das �ltimas d�cadas, os principais centros produtores de startups tecnol�gicas desenvolveram uma complexa rede de elementos que contribuem significativamente para a cria��o de uma grande quantidade de startups e, por conseguinte, um bom n�vel de startups de alta qualidade e possibilidade de sucesso. Chamamos essa rede de \textbf{Ecossistema de Startups} e, nesta se��o, apresentamos os principais elementos que o comp�em e seus inter-relacionamentos. A Figura \ref{fig:ecossistema} apresenta um \emph{arcabou�o conceitual} que descreve os atores e for�as presentes nesse ecossistema.


\begin{figure}[htbp]
\centerline{\includegraphics*[angle=90,width=12cm]{GeneralizedFramework}}
\caption{Arcabou�o conceitual do ecossistema de startups (a traduzir)}
\label{fig:ecossistema}
%\vspace{-5mm}%para eliminar espa�o excessivo
\end{figure}


\subsection{Centros empreendedores}
respons�vel: Fabio revisor:



\subsection{Perfil do empreendedor e de times fundadores}
respons�vel: Fabio revisor:

\subsection{Incubadoras e aceleradoras}
respons�vel: Fabio revisor:

\subsection{Financiamento}
respons�vel: Fabio revisor:

\subsection{Propriedade Intelectual, Patentes e Software Livre}
respons�vel: Fabio revisor:

\subsection{O cen�rio Brasileiro}
respons�vel: Fabio revisor:

\section{Conclus�es}
respons�vel: Fabio revisor:
