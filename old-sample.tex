\section{Desenvolvimento de Software Livre}

O desenvolvimento de software livre possui características distintas do
modelo fechado; a relação com o mercado, o processo de
desenvolvimento, o produto a ser oferecido devem, portanto, ser
abordados de maneiras distintas.


\subsection{\emph{A Catedral e o Bazar} e o Movimento pelo Código Aberto}

No artigo \emph{A Catedral e o Bazar} \cite{CATHEDRAL}, Eric Raymond
levanta os aspectos que contribuem para que um projeto de software
livre tenha sucesso; suas observações formaram a base do movimento pelo
código aberto (\emph{open source movement}) por ele iniciado na virada do século.
% 
Neste artigo, Raymond identifica que alguns dos primeiros projetos de software livre
de sucesso, como o núcleo do Emacs desenvolvido por Richard Stallman, por exemplo, utilizavam 
um modelo semelhante àquele utilizado para a construção de catedrais góticas e também usado 
largamente no desenvolvimento de software restrito.
%
Software era desenvolvido da mesma forma que catedrais góticas, habilmente 
criadas com extremo cuidado por arquitetos altamente qualificados ou por pequenos grupos 
desses ``magos'' trabalhando em esplêndido isolamento, com nenhuma versão \emph{beta} para 
ser liberada antes de seu tempo.

No entanto, ao observar o modelo de desenvolvimento do Linux, Raymond vislumbrou um mundo
completamente diferente, mais semelhante a um barulhento Bazar, onde centenas ou milhares de
desenvolvedores davam a sua contribuição que era, então, gerenciada por um pequeno grupo ou por
um ``ditador benevolente'' que dava um retorno aos contribuidores sobre a qualidade de seu trabalho
e aceitava ou não as contribuições na versão oficial do Linux.
%
Raymond aplicou essas práticas de desenvolvimento do modelo Bazar de forma consciente
em um novo projeto de sua autoria, o Fetchmail, e obteve relativo sucesso, o que o
motivou a escrever esse artigo. As principais características desse modelo e que
caracterizam boa parte dos projetos de software livre da atualidade são as seguintes:

\begin{itemize}

\item \textbf{Bons programas nascem de necessidades pessoais.} Um projeto tem maiores chances de sucesso
    quando o desenvolvedor principal ou grupo de desenvolvedores principais tem interesse e sentem
    a necessidade pessoal de utilizar aquele software.

\item \textbf{Bons programadores sabem escrever bom código; excelentes
    programadores sabem reescrever e reutilizar código.} Raymond menciona a ``preguiça construtiva''
    como a ideia de que não se deve reinventar a roda, mas sim reaproveitar o que já existe e,
    se for o caso, modificar o que já existe para melhorá-lo.

\item \textbf{Esteja preparado para jogar fora código-fonte se necessário e começar de novo.} 
  Dificilmente se vai acertar na primeira vez.
  
\item \textbf{Os usuários devem ser tratados como co-desenvolvedores.} Esse é o melhor caminho para 
    o aprimoramento do código e depuração eficaz.
    
\item \textbf{Libere código cedo e libere frequentemente; e ouça seus usuários.} 
  Um erro comum de pessoas e grupos que se 
  iniciam no mundo do software livre é achar que seu software ainda não está pronto para ser
  liberado, que agora ainda não é o momento certo para se fazer isso. Segundo Raymond, o quanto
  antes o código for liberado e quanto maior a frequência de liberação de novas versões, melhor
  será o retorno obtido dos usuários e a possibilidade de angariar contribuidores para o projeto.
  Não era incomum para Linus Torvalds, no início de 1991, liberar mais de uma versão do núcleo
  do Linux por dia! Isso possibilitou, naquela época, um grau de energia e motivação para o
  desenvolvimento colaborativo de software de forma distribuída nunca antes vista na história da 
  Computação.
  
\item \textbf{Dados olhos suficientes, todos erros são triviais.} Raymond chamou essa frase de Lei de Linus.
    Com milhares de pessoas lendo o código-fonte do Linux, os eventuais erros eram localizados e
    reportados muito rapidamente. Da mesma forma, com centenas de pessoas com conhecimento técnico
    para resolver aqueles erros, rapidamente aparecia um voluntário com a solução do problema.

\item \textbf{Trate seus testadores das versões Beta como um recurso valioso e eles logos tornar-se-ão
  um recurso valioso.} Não há nada mais eficaz para encontrar problemas num programa e sugerir
  melhorias em suas funcionalidades do que um grupo de usuário ativo e motivado querendo utilizar
  esse programa e testar as novas funcionalidades o quanto antes.
  
\item \textbf{A perfeição (em projetar) é alcançada não quando não há mais nada a adicionar, 
  mas quando não há nada para jogar fora.} Essa é uma ideia quase que de consenso entre os grandes
  cientistas e engenheiros. Deve-se buscar sempre as soluções mais simples.

\item \textbf{A melhor coisa depois de ter boas ideias é reconhecer as boas ideias de seus contribuidores.} 
    Às vezes, essa última é melhor. Um bom líder de um projeto de software livre não é necessariamente
    aquele que tem ótimas ideias, mas sim aquele que é capaz de criar o ecossistema de 
    colaboração que permita que as boas ideias emerjam e sejam valorizadas e adotadas.

\end{itemize}

% Paulo: Isso foi dito no histórico, pensar em retirar?
%\begin{comment}
Sete meses após a publicação do artigo de Eric Raymond, fortemente influenciada
pelas ideias apresentadas, a Netscape Communications, Inc. anunciou seus planos de
abrir o código do seu navegador Web. 
%
Essa iniciativa, posteriormente, levou ao desenvolvimento do Mozilla Firefox,
utilizado hoje em dia por centenas de milhões de internautas em todo o globo.
%\end{comment}


\subsection{Interação com a Comunidade}

Para haver interesse econômico em trabalhar com software livre, é
essencial aproveitar os recursos de que a comunidade dispõe, fomentando
o compartilhamento de ideias.
%
Isso envolve administrar uma equipe de desenvolvimento onde não há
hierarquia formal, não há mecanismos de pressão para o cumprimento de
prazos e não há grande formalismo em processos.
%
Além disso, os aspectos pessoais que garantem o envolvimento da
comunidade devem ser observados e respeitados.


Considerando a produção de código, documentação, relatos de defeitos, entre
outras atividades relacionadas, as comunidades de software livre vem proporcionando
a construção coletiva de sistemas de software reconhecidamente de qualidade, em
um ambiente de colaboração constante para atualização e evolução desses
sistemas, organizados na forma de um rossio \cite{simon:08}.
%
Nesse contexto, os usuários não necessariamente restringem-se a serem apenas
agentes passivos, mas podem atuar como colaboradores ou produtores do software
que usam.
%
Esse fenômeno de produção coletiva transbordou o movimento do software
livre, com maior força na primeira década do século 21, com surgimento de 
serviços criados e mantidos pelos próprios usuários na Internet, como a
Wikipedia, YouTube, blogs pessoais, TVs e rádios online, o que somado às redes
sociais como Facebook, Orkut e Twitter, fazem com que as pessoas realmente
acreditem que podem influenciar outras através de seus próprios meios de
comunicação~\cite{castells:06}.
%
Esse cenário, em que não fica clara uma diferenciação entre consumidor e produtor de
informação e, no caso do software, usuário e desenvolvedor, pode ser chamado de
``cultura livre''. 

Quando uma empresa quer interagir com uma comunidade de software livre ou
criar uma entorno do seu sistema, deve ter a consciência de que existe
essa cultura livre, de tal forma que o não respeito aos aspectos culturais e
valores dessas comunidades pode decretar o fracasso da
exploração do potencial de uma comunidade.
%
As questões éticas são tão importantes quanto os atributos técnicos. A colaboração
e o retorno em forma de contribuições para comunidade é essencial.
%
As pessoas e organizações que permanecem no longo prazo colaborando de alguma maneira com o
desenvolvimento de um software livre realmente acreditam que estão fazendo a 
diferença e ajudando o mundo de alguma forma.
%
Mesmo quando as pessoas estão sendo pagas para desenvolverem, o que hoje já é uma
parte substancial dos casos, essas questões éticas e relação com outras
pessoas norteiam a participação e produtividade de um indivíduo.
%
O ponto chave é que a motivação de trabalhar em algo que a pessoa tem como seu
e de acordo com seus valores faz com o que sua dedicação seja diferenciada.


Com o modelo de negócio baseado na prestação serviços, muitas empresas 
desenvolvem ou colaboram com um software livre para melhor reaproveitar
o conhecimento coletivo produzido em rede, bem como atingir numa escala
maior um possível mercado consumidor e usuários.
%
Muitos projetos de software livre são iniciados e controlados por uma única empresa,
mas que aceita colaborações externas.
%
Para isso, as empresas devem guiar seus funcionários a seguir
determinadas instruções ou métodos que incluem práticas que atraem
contribuições externas~\cite{corbucci:2011}.
%
Por outro lado, a empresa deve ter consciência de que apenas consegue manter o controle
sobre seus funcionários, pois quando não é respeitada a pluralidade, interesses
e meritocracia que envolve a comunidade, essa pode tomar o controle em
um novo projeto baseado no original (o que é conhecido como \textit{fork}).


O exemplo mais recente em que uma comunidade reorganizou-se por divergências
com a empresa que lidera o desenvolvimento de um software livre é o caso
do OpenOffice.org, em que a Oracle, um tempo depois de comprar a Sun,
tomou algumas medidas que estavam mais de acordo com o modelo do desenvolvimento
de software restrito ao invés de atender as demandas de sua comunidade.
%
Assim, a comunidade do OpenOffice.org, que também engloba outras empresas e ONGs,
não aceitou as regras e atitudes da Oracle. Portanto, criou uma nova versão do projeto
chamada LibreOffice, usando como ponto de partida do mesmo código livre do OpenOffice.org.


Não apenas empresas tentam mudar as regras e contrariam os interesses das
comunidades. Pessoas, ou seja, criadores ou líderes de projetos, também já
tiveram atitudes que forçaram a comunidade a se recriar.
%
Esse é o caso do TWiki, uma plataforma \textit{Wiki} criada em 1998 por
Peter Thoeny. Em 2008, Thoeny criou a empresa TWiki.net que passou a ter
o controle total do projeto TWiki. Dessa forma, os desenvolvedores da comunidade
abandonaram Thoeny e criaram o projeto Foswiki, um ``fork'' da versão até
então comunitária do TWiki, que até hoje segue seu desenvolvimento
independente do TWiki.
%
Tanto no caso LibreOffice quanto no Foswiki, os usuários ligados aos
movimento do software livre começaram a usá-los preferencialmente. No caso do
LibreOffice, as distribuições GNU/Linux, em especial as mais
ligadas à ética e filosofia do movimento, como o Debian, começam a trazer
na sua instalação padrão o LibreOffice ao invés do OpenOffice.org, o que determina
e influencia a mudança de mais usuários.


Entretanto, existem mais casos de sucesso do que casos problemáticos em que os projetos baseados em
comunidades de empresas se caracterizam como projetos de software livre.
%
Nas ``comunidades de empresas'' não há uma formalização entre como cada empresa
deve se dedicar ou investir no projeto. Elas formam um consórcio e tomam as
decisões estratégicas de desenvolvimento juntas.
%
Os melhores exemplos desses casos são (i) o Eclipse, com a \emph{Eclipse Foundation},
iniciada pela IBM e que hoje conta com diversas empresas parceiras, e (ii) o
Java, com o \emph{Java Community Process}, que, mesmo com a Oracle atualmente
como os direitos sobre a marca, permite que a comunidade e o consórcio
de empresas tomem as decisões sobre o desenvolvimento da linguagem~\cite{corbucci:2011}.
%


Além das questões éticas, filosóficas e políticas, os atributos técnicos
também influenciam essa relação com a comunidade, em particular no início
de um projeto. Os aspectos técnicos impactam mais na adoção por parte dos
usuários do que no número de desenvolvedores colaboradores, mas muitos
usuários podem ter o potencial de virarem futuros colaboradores~\cite{meirelles:2010}.
%
As questões técnicas do desenvolvimento de software livre passa pela qualidade
do código-fonte.
%
Do ponto de vista prático, os processos tradicionais investem
bastante tempo especificando requisitos e planejando a arquitetura do software para
que o trabalho possa ser dividido entre os programadores, cuja função é
transformar a análise, anteriormente concebida, em código.
%
Novas metodologias, por outro lado, como as tipicamente utilizadas no
desenvolvimento de software livre, são orientadas pela entrega constante de
valor ao ``cliente'' (comunidade e usuário final).

Nesse último caso, se houver um agente financiador ou uma comunidade de software livre como
cliente, o foco está na entrega de funcionalidades que possam ser rapidamente
colocadas no ambiente de produção, assim como em receber contínuas avaliações dos usuários para
que a equipe de desenvolvimento, a cada iteração, atenda cada vez mais as
necessidades desses ``clientes''.
%
Seja qual for a metodologia, formal ou não, adotada por uma comunidade de software
livre, ela deve ser norteada por dois aspectos fundamentais para
o controle da qualidade do projeto e facilidade em receber contribuições
no longo prazo: a qualidade do código-fonte e a qualidade dos testes.


O código-fonte de uma aplicação é tipicamente desenvolvido gradativamente e
diferentes programadores fazem alterações e extensões continuamente.
%
Portanto, novas funcionalidades são adicionadas e falhas sanadas ao longo das iterações
de desenvolvimento e manutenção do software.
%
Um aspecto importante desse processo é que há uma diferença significativa na
proporção entre linhas de código lidas e inseridas por um
programador~\cite{martin:2008,beck:2007}.
%
Dessa forma, a qualidade do código pode influenciar o crescimento da comunidade, assim como,
uma comunidade ativa pode influenciar na qualidade do código produzido.
%
Uma das estratégias de atrair contribuidores, ou evitar que eles saiam,  é ter
um código de qualidade (por exemplo, baixa complexidade estrutural e modularizado),
o que vai gerar um ciclo positivo na questão da qualidade do
código~\cite{meirelles:2010}.
%
Um estudo recente mostra que desenvolvedores antigos e centrais do projeto
inserem menos complexidade e diminuem a complexidade dos módulos (ou classes)
que contribuem. Assim, o tempo de colaboração de um desenvolvedor também
influencia na qualidade do código~\cite{terceiro:2010}.
%
Em suma, qualidade é algo subjetivo, mas muitos pontos, baseado nas experiências
dos desenvolvedores e das comunidades, podem ser tornar (ou já são) consenso, 
podendo ser mapeados de forma objetiva para métricas de código-fonte.
%
Por exemplo, os conceitos de código limpo ou beleza de código~\cite{martin:2008,beck:2007}
podem ser mapeados para um conjunto de métricas. Dessa forma, há possibilidade de que
essa avaliação seja semiautomática, através de um monitoramento dessas métricas.

Na busca por uma formalização de qual metodologia as empresas podem adotar
para interagir melhor com as comunidades de software livre, estudos
mostram que métodos ágeis e software livre tem formas de trabalhos semelhantes,
sendo considerado um método ágil por Martin Fowler~\cite{Fowler00orig}.
%
Um relatório técnico sobre metodologias de desenvolvimento ágil conclui que
o desenvolvimento de software livre é um método ágil \cite{Warsta2002}, bem
como os mesmos autores apontam fortes semelhanças entre métodos ágeis e
software livre em outro estudo~\cite{Warsta2003}.
%
Mais recentemente, uma ampla pesquisa sobre a comunidade ágil e do software
livre~\cite{corbucci:2011}, resultou em um mapeamento completo entre as práticas 
comuns usadas por essas comunidades de desenvolvimento de software. Conceitualmente, os
valores semelhantes são:

\begin{itemize}

\item {Indivíduos e interações são mais importantes que processos e ferramentas.}

\item {Software em funcionamento é mais importante que documentação abrangente.}

\item {Colaboração com o cliente (usuários) é mais importante que negociação de contratos.}

\item {Responder às mudanças é mais importante que seguir um plano.}

\end{itemize}

Baseado nesse mesmo estudo~\cite{corbucci:2011}, no dia-a-dia dos desenvolvedores
e equipes das comunidades de software livre, em geral, são usadas as seguintes práticas,
as quais também são parte das técnicas disseminadas nas metodologias ágeis:

\begin{itemize}

\item {Código compartilhado (coletivo).}
\item {Design simples.}
\item {Repositório único de código (SVN, Git, Bazaar, Mercurial).}
\item {Integração contínua.}
\item {Código e teste.}
\item {Desenvolvimento dirigido por testes (TDD).}
\item {Refatoração.}

\end{itemize}

Tanto do ponto de vista das empresas, que querem colaborar ou
tornar seus sistemas em software livre, quanto das pessoas que iniciam ou
liberam um projeto como software livre, é essencial ter uma ideia clara
desses valores e utilizar tais práticas e técnicas.
%
Na basta apenas liberar o código, uma sistematização de como interagir ou
criar uma comunidade é importante, pois dentro do universo de projetos de
software livre há uma concorrência natural.
%
Isso é semelhante à teoria da evolução, onde os mais
fracos e menos adaptados foram eliminados ao longo do tempo. Muitos
projetos de software livre ficam pelo caminho, o que significa que,
apesar de todo sucesso desse fenômeno, as comunidades, inclusive com
ajuda de empresas, podem se organizar melhor.


Quando comparamos apenas os projetos de software livre, a quantidade de projetos inativos
é alta em relação ao número de projetos considerados ativos.
%
Para ilustrar esse cenário, podemos observar os dados que extraímos, em novembro de 2009, do 
SourceForge.net, um dos mais populares repositórios de projetos de software livre.
Entre os seus 201.494 projetos cadastrados, apenas 60.642
tem mais de um lançamento (\textit{release}), 40.228 tem mais de um download e 23.754 tem
mais de um membro.
%
Em suma, isso implica em apenas 12.141 projetos quando consideramos esses 3
critérios de seleção juntos. Isso pode significar que não mais que 6\% dos
projetos no SourceForge.net são capazes de constituir uma comunidade de usuários
e desenvolvedores que se beneficiem do estilo de desenvolvimento
Bazaar~\cite{CATHEDRAL}.
%
Isso expõe que há muitas iniciativas e tentativas que ficam pelo caminho. Por outro,
lado mostra que há uma grande disposição das pessoas em criarem projetos de software
livre. Assim, há muito espaço para as empresas promoverem oportunidades para que
essas pessoas colaborem com o desenvolvimento de software, de tal forma que
empresas e comunidades juntas possam tirar proveito dessa colaboração, de
acordo com esse estilo de desenvolvimento de software.

\begin{comment}
    \begin{itemize}

    \item As questões éticas são importantes para a comunidade.

    \item As questões técnicas \emph{também} são importantes para a
    comunidade.

    \item Software Livre X software gratuito: o impacto do Java,
    Flash, Qt, drivers nVidia\ldots{}

    \item metodologias de desenvolvimento com a comunidade.

    \item ferramentas: svn, git, bazaar, mercurial.

    \end{itemize}
\end{comment}
